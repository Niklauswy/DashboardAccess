\documentclass[11pt]{article}
\usepackage[utf8]{inputenc} % Soporte para caracteres especiales
\usepackage{titlesec}       % Mejor control de títulos
\usepackage{geometry}       % Control de márgenes de página
\usepackage{tikz}           % Gráficos y decoración
\usepackage{hyperref}       % Hipervínculos en PDF
\usepackage{fancyhdr}       % Control del encabezado y pie de página
\usepackage{color}          % Colores para texto

% Configuración de márgenes
\geometry{a4paper, top=2.5cm, bottom=2.5cm, left=2.5cm, right=2.5cm}

% Títulos estéticos
\titleformat{\section}{\Large\bfseries\color{blue}}{\thesection}{1em}{}
\titleformat{\subsection}{\large\bfseries\color{cyan}}{\thesubsection}{1em}{}

% Encabezados y Pie de página
\pagestyle{fancy}
\fancyhf{}
\lhead{\reportTitle}
\rhead{Manual de instalación}
\cfoot{\thepage}

% Personalización de colores
\definecolor{myTitleColor}{RGB}{40, 116, 166}

% Título principal (Portada)
\newcommand{\manualTitle}{
    \begin{center}
    \vspace*{3cm}
    {\Huge\bfseries\color{myTitleColor} Manual de Instalación}\\[1.5em]
    {\Large \textbf{\reportTitle}} \\[1em]
    {\large \today}\\[4em]
    \includegraphics[width=0.3\textwidth]{placeholder_logo.jpg}\\[4em]
    \vfill
    \textit{Preparado por: Tu Nombre}\\
    \end{center}
}

\begin{document}

% Portada
    \thispagestyle{empty} % Sin encabezado en la portada
    \manualTitle
    \newpage

% Índice (opcional)
    \tableofcontents
    \newpage

% Sección 1: Introducción
    \section{Introducción}
    Este manual describe los pasos necesarios para instalar y configurar el software. Asegúrese de contar con los requisitos previos mencionados.

% Sección 2: Requisitos previos
    \section{Requisitos previos}
    \begin{itemize}
        \item Sistema operativo: Linux, macOS, o Windows.
        \item Tener los siguientes paquetes instalados:
        \begin{itemize}
            \item \texttt{Node.js} (versión 16 o superior).
            \item \texttt{npm} (versión 8 o superior).
        \end{itemize}
        \item Conexión a internet.
    \end{itemize}

% Sección 3: Pasos para la instalación
    \section{Pasos para la instalación}
    \subsection{Descargar los archivos}
    Navegue al repositorio del proyecto en GitHub y descargue el código fuente o clonelo usando el comando:

    \begin{verbatim}
git clone https://github.com/usuario/proyecto.git
    \end{verbatim}

    \subsection{Instalar las dependencias}
    Ejecute el siguiente comando en la carpeta del proyecto para instalar las dependencias necesarias:

    \begin{verbatim}
npm install
    \end{verbatim}

    \subsection{Configurar el proyecto}
    Edite el archivo \texttt{config.js} con la información requerida:
    \begin{verbatim}
module.exports = {
    apiKey: "YOUR_API_KEY",
    port: 3000
};
    \end{verbatim}

% Sección 4: Verificación
    \section{Verificación}
    Para verificar la instalación, ejecute el siguiente comando y asegúrese de que el proyecto esté funcionando correctamente:

    \begin{verbatim}
npm start
    \end{verbatim}

    Abra un navegador y navegue a \url{http://localhost:3000}.

% Sección 5: Solución de problemas
    \section{Solución de problemas}
    Si encuentra problemas durante el proceso:
    \begin{itemize}
        \item Verifique los permisos en el sistema operativo.
        \item Asegúrese de que las versiones de Node.js y npm sean las correctas.
        \item Revise el archivo \texttt{README.md} para obtener soporte adicional.
    \end{itemize}

% Bibliografía (Opcional)
    \bibliographystyle{plain}
    \bibliography{referencias}

\end{document}