\documentclass[11pt]{article}
\usepackage{emoji}
\usepackage{listings}
\usepackage{xcolor}
\usepackage{tcolorbox}

% ========================= VARIABLES =================================
\newcommand{\reportTitle}{Manual [Windows]  }
\newcommand{\subtitle}{}

\input{/home/klaus/Latex/base/variables.tex}
\input{/home/klaus/Latex/base/packages.tex}
\input{/home/klaus/Latex/base/styles.tex}
\input{/home/klaus/Latex/base/functions.tex}

% \setlength{\parindent}{0pt}

% Define custom tcolorbox styles
\tcbset{
    note/.style={
        colback=yellow!10!white,
        colframe=yellow!50!black,
        fonttitle=\bfseries,
        title=NOTA
    },
    important/.style={
        colback=red!10!white,
        colframe=red!50!black,
        fonttitle=\bfseries,
        title=IMPORTANTE
    }
}
% Define custom colors
\definecolor{codebg}{rgb}{0.95,0.95,0.95}
\definecolor{codeframe}{rgb}{0.8,0.8,0.8}

% Customize the listings environment
\lstset{
    backgroundcolor=\color{codebg},
    frame=single,
    rulecolor=\color{codeframe},
    basicstyle=\ttfamily\small,
    keywordstyle=\color{blue},
    commentstyle=\color{gray},
    stringstyle=\color{red},
    numbers=left,
    numberstyle=\tiny\color{gray},
    stepnumber=1,
    numbersep=10pt,
    breaklines=true,
    breakatwhitespace=false,
    showspaces=false,
    showstringspaces=false,
    showtabs=false,
    tabsize=4,
    captionpos=b
}
\begin{document}
%***************** TITLE PAGE  ******************************
    \begin{titlepage}
    \newgeometry{top=2cm, bottom=3.5cm}

    \vspace*{4cm} %

    \begin{center}
        \Huge{
            \textbf{\reportTitle}
        }
        \vspace{1cm}
        \ifdefempty{\subtitle}{}{
            \Large{
                \textbf{\subtitle}
            }
        }
    \end{center}

    \vfill

    \begin{center}
        \large{Ensenada, Baja California}
    \end{center}
\end{titlepage}

    \newgeometry{top=3cm, bottom=3cm}

    \section{Introducción}

    En este manual, se describen los pasos necesarios para configurar y unir una máquina Windows a un dominio Active Directory. Se detallan las configuraciones de red, la instalación de paquetes necesarios, y los comandos para verificar y unirse al dominio. Este documento está dirigido a administradores de sistemas y usuarios avanzados que necesiten integrar sus sistemas Windows en un entorno de dominio Active Directory.

    \begin{tcolorbox}[note]
        Nota: Asegúrese de tener privilegios de administrador y acceso a las credenciales del dominio antes de comenzar con las configuraciones.
    \end{tcolorbox}

    \section{Configurar IP del servidor}
    Debemos configurar la dirección IP del servidor en la configuración de red de Windows.

    \begin{enumerate}
        \item Abrir el Panel de Control.
        \item Ir a \texttt{Red e Internet} y luego a \texttt{Centro de redes y recursos compartidos}.
        \item Hacer clic en \texttt{Cambiar configuración del adaptador}.
        \item Hacer clic derecho en la conexión de red y seleccionar \texttt{Propiedades}.
        \item Seleccionar \texttt{Protocolo de Internet versión 4 (TCP/IPv4)} y hacer clic en \texttt{Propiedades}.
        \item Configurar la dirección IP y la puerta de enlace predeterminada.
    \end{enumerate}

    \section{Unirse al dominio}
    \begin{tcolorbox}[important]
        Asegurarse de que la máquina ya tenga el nombre de host (con la nomenclatura correcta) antes de unirse al dominio.
    \end{tcolorbox}
    \begin{enumerate}
        \item Abrir el Panel de Control.
        \item Ir a \texttt{Sistema y seguridad} y luego a \texttt{Sistema}.
        \item Hacer clic en \texttt{Configuración avanzada del sistema}.
        \item En la pestaña \texttt{Nombre del equipo}, hacer clic en \texttt{Cambiar}.
        \item Seleccionar \texttt{Dominio} e ingresar el nombre del dominio.
        \item Hacer clic en \texttt{Aceptar} y reiniciar la máquina si es necesario.
    \end{enumerate}

    \section{Configuraciones adicionales}
    A partir de este punto, se deben realizar configuraciones adicionales según las políticas de la organización y las necesidades específicas del entorno.

    \subsection{Configuración de políticas de grupo}
    \begin{enumerate}
        \item Abrir el Editor de directivas de grupo local (\texttt{gpedit.msc}).
        \item Configurar las políticas de grupo según las necesidades de la organización.
    \end{enumerate}

    \subsection{Configuración de DNS}
    \begin{enumerate}
        \item Abrir el Panel de Control.
        \item Ir a \texttt{Red e Internet} y luego a \texttt{Centro de redes y recursos compartidos}.
        \item Hacer clic en \texttt{Cambiar configuración del adaptador}.
        \item Hacer clic derecho en la conexión de red y seleccionar \texttt{Propiedades}.
        \item Seleccionar \texttt{Protocolo de Internet versión 4 (TCP/IPv4)} y hacer clic en \texttt{Propiedades}.
        \item Configurar los servidores DNS según las necesidades de la organización.
    \end{enumerate}

    \bibliography{/home/klaus/Latex/base/main}
    \bibliographystyle{plain}

\end{document}