% Template para Practicas/Tareas/Actividades
% latex -file-line-error -interaction=nonstopmode main.tex
% bibtex main
\documentclass[11pt]{article}
\usepackage{emoji}
\usepackage{listings}
\usepackage{xcolor}
\usepackage{tcolorbox}

% ========================= VARIABLES =================================
\newcommand{\reportTitle}{Manual [Linux]  }
\newcommand{\subtitle}{}

\input{/home/klaus/Latex/base/variables.tex}
\input{/home/klaus/Latex/base/packages.tex}
\input{/home/klaus/Latex/base/styles.tex}
\input{/home/klaus/Latex/base/functions.tex}

% \setlength{\parindent}{0pt}


% Define custom tcolorbox styles
\tcbset{
    note/.style={
        colback=yellow!10!white,
        colframe=yellow!50!black,
        fonttitle=\bfseries,
        title=NOTA
    },
    important/.style={
        colback=red!10!white,
        colframe=red!50!black,
        fonttitle=\bfseries,
        title=IMPORTANTE
    }
}
% Define custom colors
\definecolor{codebg}{rgb}{0.95,0.95,0.95}
\definecolor{codeframe}{rgb}{0.8,0.8,0.8}

% Customize the listings environment
\lstset{
    backgroundcolor=\color{codebg},
    frame=single,
    rulecolor=\color{codeframe},
    basicstyle=\ttfamily\small,
    keywordstyle=\color{blue},
    commentstyle=\color{gray},
    stringstyle=\color{red},
    numbers=left,
    numberstyle=\tiny\color{gray},
    stepnumber=1,
    numbersep=10pt,
    breaklines=true,
    breakatwhitespace=false,
    showspaces=false,
    showstringspaces=false,
    showtabs=false,
    tabsize=4,
    captionpos=b
}
\begin{document}
%***************** TITLE PAGE  ******************************
    \begin{titlepage}
    \newgeometry{top=2cm, bottom=3.5cm}

    \vspace*{4cm} %

    \begin{center}
        \Huge{
            \textbf{\reportTitle}
        }
        \vspace{1cm}
        \ifdefempty{\subtitle}{}{
            \Large{
                \textbf{\subtitle}
            }
        }
    \end{center}

    \vfill

    \begin{center}
        \large{Ensenada, Baja California}
    \end{center}
\end{titlepage}

    \newgeometry{top=3cm, bottom=3cm}

    \section{Introducción}

    En este manual, se describen los pasos necesarios para configurar y unir una máquina Linux a un dominio Active Directory. Se detallan las configuraciones de red, la instalación de paquetes necesarios, y los comandos para verificar y unirse al dominio. Además, se incluyen configuraciones adicionales específicas para diferentes distribuciones de Linux como Ubuntu, Debian y Raspberry Pi OS. Este documento está dirigido a administradores de sistemas y usuarios avanzados que necesiten integrar sus sistemas Linux en un entorno de dominio Active Directory.

    \begin{tcolorbox}[note]
        Nota: Asegúrese de tener privilegios de administrador y acceso a las credenciales del dominio antes de comenzar con las configuraciones.
    \end{tcolorbox}

    \section{Congigurar IP servidor}
    Debemos configurar el archivo /etc/hosts para que la máquina pueda resolver el nombre del servidor.
    \begin{lstlisting}
        sudo nano /etc/hosts
    \end{lstlisting}
    Añadir la siguiente linea al archivo.
    \begin{lstlisting}
        148.231.211.158 server
        148.231.211.158 access
    \end{lstlisting}



    \section{Instalación de paquetes}
    Para que funcione el active directoy en linux, se necesitar varios paquetes, entre ellos \texttt{realmd} y \texttt{sssd}.
    Pero bastará con instalar \texttt{realmd} para que se instalen los paquetes necesarios.

    \begin{lstlisting}
        sudo apt install realmd
    \end{lstlisting}






    \section{Unirse al dominio}
    \begin{tcolorbox}[important]
        Asegurarse de que la máquina ya tenga el hostname (con la nomeclatura correcta)  antes de unirse al dominio.
    \end{tcolorbox}
    \begin{enumerate}
        \item Verificar dominio. \\
       Ejecutar el siguiente comando para verificar el dominio este funcionado.
        \begin{lstlisting}
    sudo realm discover domain
        \end{lstlisting}

        \item Unirse al dominio \\
        Para unirse al dominio, se necesita un usuario con permisos de administrador.
        \begin{lstlisting}
    sudo realm join domain -V
        \end{lstlisting}
    \end{enumerate}


    \section{Configuraciónes adicionales}
    A partir de este punto, se deben realizar configuraciones adicionales, pero dependen de la distribución de linux que se este utilizando. A continuación se muestran algunas configuraciones en 3 distribuciones populares.

    \subsection{Ubuntu [22.] }


    \subsection{Debian}

    \subsection{Raspberry Pi OS}
    \bibliography{/home/klaus/Latex/base/main}
    \bibliographystyle{plain}

\end{document}
